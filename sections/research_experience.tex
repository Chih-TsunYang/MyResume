%----------------------------------------------------------------------------------------
%	SECTION TITLE
%----------------------------------------------------------------------------------------
\cvsection{Research \& Working Experience}
%----------------------------------------------------------------------------------------
%	SECTION CONTENT
%----------------------------------------------------------------------------------------
\begin{cventries}
\cventry
{Prof. Chin-Hui Yu's Lab, Department of Chemistry, NTHU} % Institution
{Undergraduate Researcher \& Research Assistant}
{Hsinchu, Taiwan} % Location
{Jan. 2017 -- Mar. 2019}
{
\begin{minipage}{\textwidth}
\textbf{Research Project Topic:}\\
\textit{"Using Constrained Multi-Coordinate Driven Method to Study the Regioselectivity in Hydroxybenzyl Alcohol Formation Under Alkaline Aqueous Environments"}
\begin{itemize}
    \item Using the constrained reduced dimensionality (CRD) algorithm, combined with Gaussian 09, to study how phenol and formaldehyde form hydroxybenzyl alcohol in alkaline aqueous environment. 
\end{itemize}
\textbf{Computational Cluster \& Network Management:}
\begin{itemize}
    \item Managed and maintained the Linux-based computer cluster(40+ PCs) in our lab.
    \item Successfully compiled and deployed Gaussian 16 in our computational cluster.
\end{itemize}
\end{minipage}
}

%------------------------------------------------

\cventry
{Prof. Li-Kang Chu's Lab, Department of Chemistry, NTHU}
{Master's Student}
{Hsinchu, Taiwan}
{Sep. 2019 -- Aug. 2021}
{
\begin{minipage}{\textwidth}
\textbf{Thesis Topic:}\\
\textit{"Investigating the Protein Dynamics of Human Serum Albumin in Hypothermic and Normothermic Conditions with Temperature Jump"}
\begin{itemize}
    \item Developed a tryptophan-based fluorescence temperature jump system to study the thermally-induced dynamic process of human serum albumin.
\end{itemize}
\textbf{Side Project 1: Nanosecond Transient Absorption Spectrometer}
\begin{itemize}
    \item Constructed a nanosecond transient absorption Spectrometer that uses ICCD as the detector.
    \item Programmed a script in PI's WinSpec to automatically acquire spectrum.
\end{itemize}
\textbf{Side Project 2: Upgrade the Stopped-flow Apparatus in the Teaching Laboratory}
\begin{itemize}
    \item Repaired an old stopped-flow apparatus by replacing the ADC card and electronic components.
    \item Developed a new software in Java, including the data collecting routine and the user interface.
\end{itemize}
\end{minipage}
}

%------------------------------------------------
\cventry
{Dr. Kuo-Hua Huang's Lab, Institute of Molecular Biology, Academia Sinica}
{Research Assistant}
{Taipei, Taiwan}
{Sep. 2021 -- Aug. 2022}
{
\begin{minipage}{\textwidth}
\textbf{Research Project Topic:}\\
\textit{"Constructing Virtual Animal Models and Virtual Reality Environment to Allow Future Study on the Roles of Reciprocal Interaction in Zebrafish Social Behavior"}
\begin{itemize}
    \item Constructed a zebrafish 3D model and animated it in Blender.
    \item Developed a close-loop VR, including image acquisition, real-time image processing, and scene updating in Python \& Unity.
    \item Utilized Python to analyze the behavior motif of freely-swimming zebrafish to make the animation more realistic.
\end{itemize}
\end{minipage}
}
\end{cventries}
